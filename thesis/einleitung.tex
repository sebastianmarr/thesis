\chapter{Einleitung}

Immer mehr Online--Plattformen geben ihren Benutzern die Möglichkeit, sich an der Beschreibung, Bewertung und Kategorisierung von Inhalten zu beteiligen. Zu diesen Beteiligungsmöglichkeiten gehören beispielsweise die Vergabe von Tags, Produktbewertungen in Online--Shops oder Kommentarfunktionen in Blogs und auf Nachrichtenseiten. 

Speziell Tagging--Systeme bieten ein großes Potenzial, die Organisation von Inhalten auf Websites nachhaltig zu verändern \cite{sc2005}. Sie erlauben den Benutzern einer Website, Inhalte mit Begriffen zu versehen, um diese zu beschreiben oder zu kategorisieren. Die Eingabe der Tags unterliegt dabei möglichst wenigen Regeln, um dem Benutzer zu ermöglichen, den Inhalt in einer für ihn natürlichen Art zu beschreiben. Die Benutzer bauen dabei eine Begriffswelt auf, die ihre Sicht auf die Inhalte der Website beschreibt.

Diese Begriffswelt besteht in einer losen Ansammlung von Begriffen und deren Verknüpfung mit Inhalten. Jedoch liegt die Vermutung nahe, dass auch zwischen den Begriffen selbst Zusammenhänge existieren. Diese werden zwar von den Benutzern nicht explizit in das System eingegeben, jedoch bei der Vergabe von Tags bedacht.

Das Finden dieser Zusammenhänge bietet einige Nutzungsmöglichkeiten, die die Benutzererfahrung auf der Website verbessern können. Denkbar sind beispielsweise Navigationsstrukturen, die diese Zusammenhänge berücksichtigen, um den Benutzer zu für ihn relevanten Inhalten zu führen. Auch die Suchfunktion einer Website kann maßgeblich verbessert werden, wenn zu einem Suchbegriff weitere Begriffe bekannt sind, die den Suchraum erweitern.

Die Suche nach Zusammenhängen in einer gegebenen Datenmenge wird als \emph{Link Discovery} bezeichnet. Die Methoden, die dazu angewendet werden, hängen stark von der Art der Daten und der geplanten Anwendung ab. Für Tagging--Daten bietet es sich an, Kookkurrenzen zu ermitteln. Diese ergeben sich aus der Verwendung von mehreren Tags zur Beschreibung des gleichen Inhaltes. Werden Tags häufig zusammen verwendet, besteht eine hohe Wahrscheinlichkeit, dass zwischen diesen ein Zusammenhang besteht. 

In den Arbeiten von \textcite{ps2006} und \textcite{kss2010} wurde dieser Ansatz, angewendet auf Tagging--Systeme, beschrieben. Diese konzentrierten sich auf die Ableitung von Themenclustern aus den ermittelten Beziehungen. In der Arbeit von \textcite{ps2006} wird ein Ansatz beschrieben, eine Ontologie aus den Daten der Foto--Plattform \emph{Flickr} herzustellen. Dazu wird versucht, ebenfalls auf Basis von Kookkurrenz, hierarchische Beziehungen zwischen den Begriffen zu ermitteln.

In der vorliegenden Arbeit wird ebenfalls ein kookkurrenzbasierter Ansatz beschrieben, Zusammenhänge zwischen den Begriffen eines Tagging--Systems herzustellen. Dabei wird ein Weltausschnitt erstellt, der die Begriffe, den Kontext ihrer Verwendung und die Beziehungen zwischen den Begriffen enthält. Um die Nutzbarkeit dieses Weltausschnittes weiter zu verbessern, werden zusätzliche interne und externe Datenquellen integriert, um weitere Kontexte der Begriffe zu erhalten. Daraus ergeben sich Zusammenhänge verschiedener Typen. Um eine nach Relevanz geordnete Liste von Beziehungen eines Begriffes zu erhalten, wird ein Priorisierungsverfahren vorgestellt, das diese Typen gegeneinander gewichtet. Das Vorgehen wird dabei am Beispiel von Produkttags der E--Commerce--Plattform Spreadshirt demonstriert.

\section{Zielsetzung der Arbeit}

Das Ziel dieser Arbeit besteht in der Herstellung von Beziehungen zwischen Begriffen, der so genannten Link Discovery. Ausgangspunkt dafür sind die Daten eines Tagging--Systems. Dazu wird zuerst ein Framework definiert, welches die theoretischen Grundlagen der Link Discovery beschreibt. Dieses Framework wird zur Durchführung der Link Discovery an konkreten Daten verwendet und die Ergebnisse präsentiert. Die inhaltliche Qualität der Beziehungen hängt stark von den verwendeten Daten und der geplanten Anwendung ab und liegt nicht im Fokus dieser Arbeit. Außerdem werden die Anforderungen an ein System, das die Link Discovery technisch umsetzt, formuliert und deren Realisierung diskutiert.

Im Detail werden die folgenden Fragen beantwortet:

\begin{itemize}
    \item Was sind Tagging--Systeme, welche Arten von Tagging--System gibt es und welche besonderen Eigenschaften hat das beispielhaft verwendete System des Unternehmens Spreadshirt?
    \item Was ist Link Discovery und wie kann diese umgesetzt werden, um Beziehungen aus Produkttags zu extrahieren?
    \item Was ist Kookkurrenz und wie kann diese zur Link Discovery genutzt werden?
    \item Wie kann eine Graphenrepräsentation genutzt werden, um die Ergebnisse der Link Discovery abzubilden?
    \item Wie können die erzeugten Beziehungen durch Data Mining oder Integration weiterer Datenquellen angereichert werden und welche Datenquellen sind dazu geeignet?
    \item Wie können die erzeugten Beziehungen mittels interaktiver evolutionärer Algorithmen priorisiert werden, um für einen Anwendungsfall relevante Nachbarn eines Begriffes zu erhalten?
    \item Welche technischen Anforderungen stellt die Link Discovery und wie kann die Berechnung von Beziehungen implementiert und beschleunigt werden?
\end{itemize}

\section{Aufbau der Arbeit}

Die Ergebnisse dieser Arbeit werden in vier Hauptkapiteln vorgestellt. Zunächst werden in \cref{tagging} Tagging--Systeme im Allgemeinen und das beispielhaft verwendete System des Unternehmens Spreadshirt im Speziellen erläutert und die Eigenschaften, Datenqualität und Menge der Daten diskutiert.

In \cref{ld_framework} werden mit dem Link--Discovery--Framework die theoretischen Grundlagen für die spätere Durchführung der Link Discovery definiert. Dies umfasst die Modellierung des betrachteten Weltausschnittes in \cref{world_model}, die Beschreibung des Vorgehens in \cref{ld_process}, Kookkurrenz als Mittel zur Beziehungserzeugung in \cref{co-occurence}, die Überführung des Weltausschnittes in eine Graphenrepräsentation in \cref{graphs}, mögliche Datenquellen zur Anreicherung in \cref{data_sources} und die Einführung evolutionärer Algorithmen zur Priorisierung von Beziehungen in \cref{evo_for_prio}.

\cref{link_discovery} beschäftigt sich mit der Umsetzung des in \cref{ld_framework} beschriebenen Frameworks an konkreten Daten. Dazu wird in \cref{ld_tags} die initiale Erstellung aus den Daten des Tagging--Systems von Spreadshirt beschrieben. Diese Daten werden in \cref{clicktracking} mit den Daten des Clicktracking--Systems von Spreadshirt angereichert. In \cref{decomposition} wird die weitere Anreicherung durch Zerlegung von Wortgruppen beschrieben. Ein letzter Anreicherungsschritt besteht in der in \cref{wortschatz} beschriebenen Integration der Daten des Wortschatzes der Universität Leipzig. Die quantitativen Ergebnisse dieser Schritte werden in \cref{lda_results} ausgewertet. In \cref{lda_prio} wird die konkrete Priorisierung der erzeugten Beziehungen erläutert und die Ergebnisse diskutiert.

Aspekte der technischen Umsetzung des Link--Discovery--Systems werden in \cref{system} beschrieben. Dazu gehören die Formulierung der Anforderungen an ein solches System in \cref{requirements}, die Beschreibung der Systemarchitektur in \cref{architecture} sowie die erfolgte Technologieauswahl in \cref{tech}.

Den Abschluss der Arbeit bildet \cref{summary} mit einer Zusammenfassung der Ergebnisse und einem Ausblick auf mögliche weitere Arbeiten.

In dieser Arbeit wurden die Diagrammtypen der Methodik FMC \cite{fmc} zur Modellierung von Architektur, Prozessen und Daten verwendet. Code--Beispiele für Beispieldaten sind in JSON--Notation \cite{json2006} formuliert.

Der Quelltext dieser Arbeit und der implementierten Link--Discovery--Schritte wurden unter \url{http://github.com/sebastianmarr/thesis} veröffentlicht.