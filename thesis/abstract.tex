\begin{abstract}
In vielen Online-Shops und Marktplätzen können die angebotenen Artikel von Benutzern mit Tags versehen werden. Diese Tags stellen einen Zusammenhang zwischen dem angebotenen Artikel und einem inhaltlichen Thema her. Die Auswertung und Nutzung dieser Zusammenhänge kann das Einkaufserlebnis der Kunden maßgeblich verbessern und somit einen wirtschaftlichen Nutzen für den Betreiber des Online-Shops bedeuten. Diese Masterarbeit beschäftigt sich mit der Auswertung von Produkt-Tags. Mit Hilfe eines Kookkurrenzgraphen und Integration von externen Daten wie Wörterbüchern und Clicktracking-Daten werden inhaltliche Beziehungen zwischen Bergiffen hergestellt. Hauptaugenmerk liegt dabei auf der Verbesserung der Datenqualität, um die Beziehungen in Hinblick auf die möglichen Anwendungsmöglichkeiten zu verbessern. Dabei werden die grundsätzlichen Vorgehensweisen zur Bereinigung, Integration, Reduktion, und Transformation in eine Graph-Datenstruktur beschrieben und angewendet. Anschließend wird eine Optimierung der verwendeten Parameter durchgeführt und die Ergebnisse evaluiert.
\end{abstract}