\begin{abstract}
In vielen Online-Shops und Marktplätzen können die angebotenen Artikel von Benutzern mit Tags versehen werden. Diese Tags stellen einen Zusammenhang zwischen dem angebotenen Artikel und einem inhaltlichen Thema her. Die Auswertung und Nutzung dieser Zusammenhänge kann das Einkaufserlebnis der Kunden maßgeblich verbessern und somit einen wirtschaftlichen Nutzen für den Betreiber des Online-Shops bedeuten. Diese Masterarbeit beschäftigt sich mit der Auswertung von Produkt-Tags. Mit Hilfe eines Kookkurrenz-Graphen und Integration von externen Daten wie Wörtbüchern und Clicktracking-Daten werden Cluster von Begriffen erstellt, die thematische Zusammenhänge darstellen. Hauptaugenmerk liegt dabei auf der Verbesserung der Datenqualität, um das Clusteringergebnis zu verbessern. Dabei werden die grundsätzlichen Vorgehensweisen zur Bereinigung, Integration, Reduktion, und Transformation in eine Graph-Datenstruktur beschrieben und angewendet. Anschließend wird Auswahl des Clustering-Algorithmus, dessen Vorgehensweise und die Auswertung der Ergebnisse beschrieben und durchgeführt.
\end{abstract}