\chapter{Link Discovery}

Um den erstellten Graphen aufzuwerten, bietet sich die Integration externer Datenquellen an. Das folgende Kapitel beschreibt die im Rahmen dieser Arbeit verwendeten Datenquellen und deren Integration in den Graphen näher. Konkret handelt es sich dabei um die Google Translate API zur automatischen Spracherkennung sowie das Wortschatz-Projekt der Universität Leipzig zur Ingration von Wortbedeutungen und lingustischen Beziehungen zwischen Begriffen.
\section{Tags}

\section{Clicktracking}

\section{Google Translate}

Google bietet im Rahmen seines \emph{Translate}-Services \cite{gt2013} eine kostenpfichtige API für Spracherkennung an. Diese ermöglicht es, die Sprache beliebiger Zeichenketten automatisch erkennen zu lassen. Google stellt hierzu eine REST-API zur Verfügung.

Diese Schnittstelle liefert Ergebnisse der Form \((l, c)\), wobei \(l\) die für die Zeichenkette erkannte Sprache und \(c\) einen Konfidenzwert für die Spracherkennung repräsentiert. Der Konfidenzwert liegt dabei im Intervall zwischen \num{0} und \num{1} und stellt die Verlässlichkeit der Spracherkennung dar.

Die Integration der Spracherkennungs-Daten in den Graphen gestaltet sich einfach. Dazu werden die durch die bereits vorhandenen Knoten repräsentierten Zeichenketten extrahiert und als Eingabedaten für die Spracherkennungs-API verwendet. Die Ergebnisse werden abgespeichert, um weitere kostenpflichtige Abfragen zu vermeiden.

Eine Bereinigung der Ergebnisse ist nicht erforderlich. Somit müssen die Ergebnisse lediglich in den Ausgangsgraphen integriert werden. Die Spracherkennung an sich bringt keine Ähnlichkeitsbeziehungen mit sich, sondern verbessert gegebenenfalls nur die Knotenauswahl für spätere Operationen.

In der Konsequenz genügt es also, die für die Abfrage verwendeten Knoten mit den Ergbenissen der Spracherkennung zu annotieren. Somit kann dann bei späteren Analysen anhand des Konfidenzwertes abgewogen werden, ob die erkannte Sprache oder die eventuell schon am Knoten vorhandene Sprache verwendet werden soll.

\section{Zerlegung von Wortgruppen}

\section{Wortschatz der Universität Leipzig}

Die Universität Leipzig betreibt ein Wortschatz-Projekt \cite{ws2013}. Im Rahmen dieses Projektes wird durch die Analyse von großen Textmengen eine Datenbank deutscher Wörter, deren Bedeutungen, grammatikalische Eigenschaften und Beziehungen zu anderen Wörtern aufgebaut.