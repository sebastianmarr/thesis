\chapter{Einleitung}

Mit der steigenden Menge von nutzergenerierten Inhalten steigt auch die Menge von Metadaten, die mit diesen Inhalten verknüpft sind. Zur späteren Durchsuchbarkeit und Kategorisierung geben viele Online-Plattformen, Marktplätze und Online-Shops ihren Benutzern die Möglichkeit, Inhalte mit Metadaten zu versehen.

Eine oft genutzte Möglichkeit zur Beschreibung von Inhalten sind Tags. Dabei handelt es sich um Wörter oder Wortgruppen, die vom Benutzer frei gewählt werden können, um den Inhalt zu beschreiben. Dabei unterliegt die Eingabe von Tags möglichst wenigen Regeln, um dem Benutzer eine für ihn natürliche Beschreibung des Inhaltes zu ermöglichen. Dabei ist explizit, im Gegensatz zu einer Kategorisierung, die Vergabe von mehreren Tags vorgesehen.

Ein charakteristisches Merkmal von Tags ist dabei, dass sie nur einen bestimmten Aspekt des getaggten Objektes beschreiben. Dabei sind Tags nicht hierarchisch und es werden an keiner Stelle vom Nutzer explizite Zusammenhänge zwischen Tags erstellt. Jedoch liegt die Annahme, dass zwischen Tags Beziehungen herstellbar sind und sich mehrere Tags zu übergeordneten Themen zusammenfassen lassen, nahe. Der Benutzer berücksichtigt diese Beziehungen bei der Eingabe des Tags, formuliert sie jedoch nicht explizit. Die nachträgliche Rekonstruktion der Denkprozesse bei der Eingabe von Tags ist Thema dieser Arbeit.

Dabei ist zu beachten, dass Benutzer bei der Eingabe von Tags unterschiedliche Ziele verfolgen. Idealerweise werden Tags so vergeben, dass sie das getaggte Objekt inhaltlich beschreiben. Jedoch werden vom Benutzer bei der eingabe des Tags weitere Assoziationen hergestellt. Beispielsweise kann der Benutzer mit einem Objekt bestimmte Emotionen oder Wertungen verbinden, die sich in den vergebenen Tags wiederspiegeln.

Auf Marktplätzen, bei denen Verkäufern die Möglichkeit des Taggings ihre Produkte gegeben wird, besteht eine weitere Motivation in der Erhöhung der Auffindbarkeit des Produktes. Dabei kann die inhaltliche Qualität des Tags außer Acht gelassen werden, wenn bei Vergabe einer falschen Beschreibung die Sichtbarkeit des Produktes erhöht wird. Auch der Marktplatzbetreiber selbst kann so vorgehen, um zu versuchen, die Gesamtverkäufe zu steigern.

Die verschiedenen Motivationen der Benutzern von Tags erschwerden die nachträgliche Suche nach Assoziationen. Die vorliegende Masterarbeit beschäftigt sich mit Strategien zur Datenaufbereitung und Nutzung externer und interner Datenquellen und deren Integration mit Tag-Daten. Aus diesen Datenquellen wird eine Datenstruktur mit einem Kookurrenzgraphen als Basis aufgebaut und schließlich werden mit Hilfe von Clustering-Algrithmen daraus Themen extrahiert. Außerdem wird eine Evaluation der Ergebnisse und eine Analyse der verwendeten Methoden durchgeführt.

\section{Motivation und Anwendungen}

Die nachträgliche Herstellung von Assoziationen in vorhandenen Tag-Daten bietet einige Nutzungsmöglichkeiten für den Betreiber der Online-Plattform.

Die Beziehungen zwischen Tags können genutzt werden, um Suchergebnisse zu verbessern. Wenn zu einem Suchbegiff weitere relevante Begriffe bekannt sind, können diese im Suchergebnis mit enthalten sein um somit auch Objekte zu finden, die nicht direkt mit dem Suchbegriff getaggt sind. Außerdem können Suchen vom Benutzer mit Hilfe von verwandten Tags verfeinert werden.

Über die Zusammenfassung von Tags zu Themen lässt sich außerdem die Navigation einer Webseite verbessern. Aus den Themen lassen sich Kategorien oder Hierachien von Kategorien erzeugen, die besser den Denkmustern von Benutzern entsprechen. So sind auch Navigationskonzepte denkbar, die nicht hierarchisch, sondern assoziativ aufgebaut sind. Desweiteren können Tag-Assoziationen für Empfehlungssysteme genutzt werden, die einem Kunden zu einem bestimmten Artikel passende andere Artikel vorschlagen.

Im Bereich des Marketings können Beziehungen zwischen Tags genutzt werden, um für bestimmte externe Suchbegriffe spezielle Seiten zu erstellen (Landing Pages), die Inhalte zu diesem Suchbegriff bereitstellen oder Werbung für diese Suchbegriffe zu schalten. Außerdem können mit Hilfe des Kaufinteresses an bestimmten Themen über die Zeit Trends erkannt werden und mit entsprechenden Marketingmaßnahmen darauf reagiert werden.

All diese Anwendungen führen zu einer besseren Erfahrung für den Benutzer der Plattform. Die Präsentation von Daten kann besser auf die Denkmuster und Erwartungshaltungen des Benutzers angepasst werden. Dies führt in Konsequenz zu einem wirtschaftlichen Vorteil für den Plattformbetreiber.

\section{Kontext}

\section{Aufbau der Arbeit}