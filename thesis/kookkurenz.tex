\chapter{Erstellung von Kookkurenzgraphen}

Das folgende Kapitel beschreibt detailliert die Erstellung von Kookkurenzgraphen. Dabei werden die Grundprinzipien erläutert, die existierenden Ähnlichkeitsmaße diskutiert und die algorithmische Umsetzung mittels MapReduce dargestellt.

\section{Grundprinzipien}

Um gewichtete inhaltliche Beziehungen zwischen Wörtern und Wortgruppen herstellen zu können, wird eine Definition von \emph{Ähnlichkeit} benötigt. Diese lässt sich auf vielfältige Arten bestimmen.

Die Ähnlichkeit zwischen zwei Dokumenten kann grundsätzlich nach \textcite{at1977} definiert werden. Dieser Definition liegt zu Grunde, dass sich die Dokumente als Mengen von Eigenschaften beschreiben lassen. Im Gegensatz zu anderen Ähnlichkeitsmodellen hängt die Ähnlichkeit nicht nur von den gemeinsamen Eigenschaften der Dokumente ab, sondern auch von den Eigenschaften, die die Dokumente allein besitzen. Somit lässt sich Ähnlichkeit \(s(A,B)\) zwischen den Dokumenten, die durch die Mengen \(A\) und \(B\) dargestellt werden, folgendermaßen definieren:
\[s(A,B) = F(A \cap B, A-B, B-A)\]
\label{similarity}

Die Gestaltung der Funktion \(s\) und die Auswahl der für die Ähnlichkeitsberechnung genutzten Eigenschaften der Dokumente hängt stark von der Anwendung ab. Somit beschreibt beispielsweise die Levenshtein-Distanz \cite{vl1966} die Ähnlichkeit zweier Zwichenketten durch die minmale Menge von Einfüge-, Lösch- und Ersetzungsoperartionen, die nötig sind, um eine Zeichenkette in die andere umzuwandeln. In Ontologien und Taxonomien kann die Ähnlichkeit von Begriffen mittels der Knoten- oder Kanteneigenschaften berechnet werden. Beispiele hierfür sind die Ähnlichkeit in Ontologien nach \textcite{pr1995} und \textcite{ps2002}. In der Bildverarbeitung können merkmalsbasierte Ähnlichkeitsmaße ebenfalls eingesetzt werden, beispielsweise beschreiben \textcite{ow2006} ein Ähnlichkeitsmaß auf Basis von Clusteringalgorithmen, die auf Rastergrafiken angewandt werden.

Im Kontext dieser Arbeit wird also ein Ähnlichkeitsmaß gesucht, dass anhand der Eigenschaften von Wörtern und Wortgruppen eine Distanz zwischen ebenjenen berechnet. Da eine inhaltliche Ähnlichkeit gesucht wird, spielen lingustische Ähnhlichkeitsmaße wie die Levenshtein-Distanz eine untergeordnete Rolle. Zu Beginn bestehen keinerlei Verbindungen zwischen den Zeichenketten, so dass keine Ähnlichkeitsmaße für Ontologien eingesetzt werden können.

Somit bietet sich die Wahl eines Ähnlichkeitsmaßes an, dass den Kontext, in dem die Wörter und Wortgruppen im Quellsystem verwednet werden, berücksichtigt.

\section{Kookkurenz}

In \ref{data} wurden die für diese Arbeit verfügbaren internen Datenquellen beschrieben. Diese haben gemeinsam, dass zu den vorhandenen Wortgruppen nur wenig Kontext verfügbar ist. Im Falle des Tag-Systems ist bekannt, an welchen Dokumenten die Tags verwendet wurden. Das Clicktracking zeichnet die zu verwendeten Suchbegriffen geklickten Dokumente auf. Beide Datenquellen liefern also die Verwendungen von Wörtern und Wortgruppen zur inhaltlichen Beschreibung von Dokumenten.

Wenn mehrere Begriffe pro Dokument verwendet werden, wird damit ein Zusammenhang zwischen den Begriffen beschrieben. Dieser Zusammenhang lässt sich mit dem Ähnlichkeitsmaß \emph{Kookkurenz} messen. Kookkurenzmaße beschreiben, wie oft Begriffe gemeinsam verwendet werden. Dies wir dabei ins Verhältnis zum einzelnen Auftreten der Begriffe gesetzt und genügt somit der Definition von Ähnlichkeit in \ref{similarity}.

Dazu muss angemerkt werden, dass die Ähnlichkeit mittels Kookkurenz nicht zwingend eine Ähnlichkeit der den Begriffen zu Grunde liegenden Konzepte darstellt. Die Verwendung von Kookkurenz als Ähnlichkeitsmaß beruht allein auf der Annahme, dass Menschen zur Beschreibung von gleichen Inhalten die gleichen begriffe benutzen. Diese Annahme muss im Laufe der Evaluation der Ergebnisse validiert werden.

\section{Kookkurenzmaße}
\label{measures}

Werden die Objekte, zwischen denen die Ähnlichkeit berechnet werden soll, als Mengen von Eigenschaften definiert, bieten sich die üblichen Kennzahlen für die Ähnlichkeiten von Mengen an. Ein Begriff kann also als Menge der Dokumente, für die er als Beschreibung verwendet wurde, definiert werden.

Um die Ähnlichkeit zwischen zwei Begriffen zu ermitteln, lassen sich die Vereinigungsmenge, Schnittmenge und Kreuzprodukte der jeweiligen Mengen bilden, die die Begriffe repräsentieren. Ist also \(A\) die Menge der Dokumente, die mit einem Begriff \(a\) versehen wurden, \(B\) die Menge der Dokumente mit einem Begriff \(b\), so ergeben sich die Mengen:

\begin{itemize}
    \item \(A \cap B\), alle Dokumente die mit \(a\) und \(b\) versehen wurden
    \item \(A \cup B\), alle Dokumente die mit \(a\) oder \(b\) versehen wurden
    \item \(A \times B\), alle Dokumentenpaare, die sich aus den Mengen \(A\) und \(B\) bilden lassen
\end{itemize}

Die Mächtigkeiten dieser Mengen können dann zur Berechnung verschiedener Ähnlichkeitsmaße verwendet werden. Drei der üblichsten Maße wurden im Rahmen dieser Arbeit verwendet und werden im folgenden genannt.

\subsection{Sørensen-Dice}

Der Sørensen-Dice-Koeffizient \cite{st1948} \cite{ld1945}, oft auch nur Dice-Koeffizient, stammt ursprünglich aus der Biologie und wurde verwendet, um die Ähnlichkeit zwischen Proben zu berechnen. Heute findet er allgemeine Anwendung im Data Mining. Er ist definiert durch:

\[
\delta_{Dice}(a, b) = \frac{2|A \cap B|}{|A|+|B|}
\]

Der Wertebereich des Koeffizienten liegt zwischen \num{0} und \num{1}.

\subsection{Jaccard}

Der Jaccard-Index \cite{pj19012} wurde ursprünglich mit dem gleichen Zweck wie der Dice-Koeffizient verwendet. Sein Wertebereich liegt ebenfalls zwischen \num{0} und \num{1} und er ist definiert durch:

\[
\delta_{Jaccard}(a,b) = \frac{|A \cap B|}{|A \cup B|}
\]

\subsection{Kosinus}

Die Kosinus-Ähnlichkeit \cite{hkp2012} ist ursprünglich ein Maß für die Ähnlichkeit zweier Vektoren. Sie ist eine Maßzahl dafür, ob die Vektoren ungefähr in die gleiche Richtung zeigen. Sie kann jedoch genauso auf Mengen angewendet werden, da das Vorhandensein der Elemente in der Menge auch durch einen Vektor in einem \(n\)-dimensionalen Raum dargestellt werden kann, wobei \(n\) die Anzahl aller möglichen Eigenschaften ist. Der Wertebereich der Kosinus-Ähnlichkeit liegt ebenfalls zwischen \num{0} und \num{1}. Sie ist auf den in \ref{measures} definierten Mengen folgendermaßen definiert:

\[
\delta_{Cosine}(a, b) = \frac{|A \cap B|}{\sqrt{|A| \times |B|}}
\]

\section{Umsetzung mit MapReduce}

\subsection{Generelle Beschreibung von MapReduce}

\subsection{Anwendung zur Ermittlung von Kookkurenzen}