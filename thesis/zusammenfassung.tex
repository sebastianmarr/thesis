\chapter{Schlussbetrachtung}
\label{summary}

In dieser Masterarbeit wurde ein Verfahren und dessen praktischeDurchführung zum Finden von Zusammenhängen zwischen Begriffen, der Link Discovery, beschrieben. Die Basis dafür stellten die Daten eines Tagging--Systems dar. Dazu wurden Tagging--Systeme grundlegend erläutert, deren Datenmodell definiert und die grundlegenden Unterschiede zwischen Folksonomies und geschlossenen Tagging--Systemen herausgearbeitet. Die Eigenschaften des im späteren Verlauf verwendeten Tagging--Systems von Spreadshirt wurden beschrieben und dessen Datenqualität in Hinblick auf Korrektheit, Vollständigkeit und Redundanzfreiheit diskutiert. Außerdem wurden die zu verarbeitenden Datenmengen definiert.

Für die Durchführung der Link Discovery wurde ein Framework definiert, welches die konzeptionelle Grundlage für die spätere Umsetzung darstellt. Dieses Framework modelliert den betrachteten Weltausschnitt, welcher Begriffe, den Kontext von Begriffen und deren Beziehungen untereinander enthält. Dieser Weltausschnitt wurde in eine Graphenrepräsentation überführt. Weiterhin wurde der Prozess der Link Discovery definiert und die einzelnen Schritte erläutert. Diese bestehen in der initialen Erstellung des Weltausschnittes, der Anreicherung durch Mining oder Integration weiterer Datenquellen und der Priorisierung der erzeugten Beziehungen. Die theoretischen Grundlagen von Kookkurrenz zur Beziehungserzeugung wurden definiert und die Berechnung veranschaulicht.

Weiterhin wurden mögliche Datenquellen diskutiert und die für die praktische Durchführung der Link Discovery in dieser Arbeit verwendeten Quellen ausgewählt. Diese bestehen aus dem Tagging-- und Clicktracking--System von Spreadshirt sowie dem Wortschatz der Universität Leipzig. Zur Priorisierung der erzeugten Beziehungen wurden evolutionäre Algorithmen erläutert und deren Einsatz im Rahmen der Priorisierung definiert.

Diese theoretischen Grundlagen wurden anschließend an konkreten Daten praktisch umgesetzt. Für jede integrierte Datenquelle wurden die Schritte Import, Bereinigung, Reduktion, Transformation in die Graphenrepräsentation und Integration in den Weltausschnitt ausführlich dargestellt und die quantitativen Ergebnisse präsentiert. Zur Anreicherung durch Mining wurde die Zerlegung von Wortgruppen in Einzelwörter erläutert. Im Anschluss wurden die quantitativen Veränderungen der Graphenrepräsentation nach jedem Link--Discovery--Schritt dargestellt und diskutiert. Nachdem alle Datenquellen integriert waren, wurde die praktische Umsetzung der Priorisierung der Beziehung mittels evolutionärer Algorithmen dargestellt und die Ergebnisse präsentiert.

Weiterhin wurden die Anforderungen an ein technisches System, welches die Link Discovery implementiert, formuliert und deren Umsetzung im Rahmen dieser Arbeit beschrieben. Dazu gehören die Wahl des Datenbanksystems MongoDB, die Implementierung der Komponenten in der Programmiersprache JavaScript und die Beschreibung von Kookkurrenzberechnung mittels des Programmiermodells MapReduce.

Die Ergebnisse dieser Arbeit stellen die Grundlage für weitere mögliche Arbeiten dar. Eine zukünftige Arbeit könnte sich mit der Auswertung der inhaltlichen Qualität der erzeugten Beziehungen beschäftigen. Dazu wird eine gründliche Analyse der integrierten Datenquellen und des Ergebnisses der Link Discovery mit Hilfe menschlicher Beurteilung benötigt.

Weiterhin sind Arbeiten denkbar, die die erzeugten Beziehungen für weitere Analyseverfahren nutzen. So könnten beispielsweise die Beziehungen zum Clustering der Begriffe zu Themen genutzt werden. Werden hierarchische Clusteringverfahren genutzt, können daraus Themenbäume und Topic Maps abgeleitet werden. Diese Themenbäume können sich durch ständige Durchführung der Link Discovery an aktuelle Trends in den Inhalten der betrachteten Website anpassen.

Der Link--Discovery--Prozess könnte durch einen interaktiven Trainingsschritt erweitert werden. In diesem Schritt wird die Beurteilung der Beziehungen durch Benutzer nicht nur zur Priorisierung, sondern zur direkten Veränderung des Weltausschnittes genutzt. Dabei werden Kanten eingefügt, die explizite statt nachträglich hergestellte Zusammenhänge beschreiben. Außerdem könnten durch manuellen Eingriff fehlerhafte Beziehungen gelöscht werden.

Insgesamt stellt das in dieser Arbeit beschriebene Framework eine gute Basis für Erweiterungen und neue Implementierungen dar. Durch Integration anderer Datenquellen und neuer Analyseverfahren kann die Qualität der gefundenen Beziehungen stetig verbessert werden. Die konkret erzeugten Daten bieten eine gute Grundlage für weitere Auswertungen und praktische Anwendungen.