\chapter{Problembeschreibung}

Das folgende Kapitel beschäftigt sich mit der Beschreibung der Problemstellung. Dazu wird zuerst das Ziel der Arbeit formuliert. Es folgen grundsätzliche Definitionen und die Beschreibung des zu bearbeitenden Datenbestandes. Abschließend wird die gewählte Lösungsstrategie konzeptionell beschrieben.

\section{Zielstellung}

Das Ziel dieser Arbeit besteht darin, aus vorhandenen Tag-Daten unter Zuhilfenahme von Integration anderer Daten Assoziationen zu extrahieren. Diese Beziehungen sollten im Optimalfall Zusammenhänge widerspiegeln, die zur Verbesserung der Benutzererfahrung beim Suchen nach bestimmten Themen, für Marketingmaßnahmen und generell für ein besseres Verständnis der auf einer Online-Plattform angebotenen Inhalte genutzt werden können.

Nutzbare Beziehungen können vielfältiger Art sein. Denkbar sind beispielsweise

\begin{itemize}
    \item inhaltliche Zusammenhänge, die mittels Clustering-Algorithmen später zu Themengebieten zusammengefasst werden
    \item Worthierarchien, aus denen Kategoriebäume erzeugt werden
    \item Wortformen, die dazu genutzt werden, mehrere Begriffe zusammenzufassen und somit mehr als nur eine wörtliche Suche zu ermöglichen
    \item Verknüpfungen von Wörtern, die über inhaltliche Zusammenhänge hinausgehen, beispielsweise Verbindungen Von Themengebieten mit bestimmten Emotionen, Produkten oder Personen
\end{itemize}

Ausgangsbasis für alle Überlegungen und Berechnungen sind die gesammelten Tag-Daten der Online-Plattform Spreadshirt, deren Struktur und Qualität im nächsten Abschnitt erläutert und diskutiert wird.

\section{Aufbau und Qualität der Daten}

Im folgenden Abschnitt sollen die intern bei Spreadshirt vorhandenen Datenquellen genannt und beschrieben werden. Außerdem wird der Umfang und die Qualität des Datenbestandes diskutiert.

\subsection{Tag-System}
\label{tag-system}

Ein Tag-System besteht im Allgemeinen aus den Mengen \(D\), \(T\) und \(U\). \(D\) bezeichnet die Menge der Dokumente. Ein Dokument \(d\) kann ein beliebiger Datensatz sein, beispielsweise ein Bild, Artikel oder Produkt. Die Menge \(U\) stellt alle Benutzer des Systems dar. Ein Benutzer \(u\) kann neben einem Index weitere Informationen besitzen, die jedoch hier im Kontext des Tag-Systems nicht tiefer gehend behandelt werden. \(T\) ist die Menge der Tags. Ein Tag \(t\) ist eine beliebige Zeichenkette. \(T\) bildet also das \emph{Vokabular} des Tag-Systems.

Die Benutzer können beliebige Dokumente mit beliebigen Tags versehen. Der Vorgang des \emph{Taggens} kann also durch die Relation \(R = D \times U \times T\) beschrieben werden, welche Tupel der Form \((d, u, t)\) enthält.

Der Betreiber der Online-Plattform kann bestimmte Aspekte des Tag-Systems begrenzen. Können alle Benutzer beliebige Tags an beliebigen Dokumente vergeben, spricht man von einer \emph{Folksonomy} \cite{ip2009}.

Im Fall von Spreadshirt ist die Vergabe von Tags auf die Menge der Partner \(P \subseteq U\) begrenzt (siehe auch \ref{spreadshirt}). Die Dokumente, die von den Partnern getaggt werden können, sind auf die Designs und Artikel beschränkt, die der Partner selbst angelegt hat. Eine Beschreibung kann also ausschließlich durch den Autor des Inhaltes erfolgen. Deshalb fehlt im Vergleich zu anderen Tag-Systemen auch die Information, welcher Benutzer den Tag vergeben hat.

Des Weiteren besitzen Tags in der Spreadshirt-Datenbank ein Attribut \emph{Sprache} aus der Menge \(L\). Die Sprache spielt bei der Eingabe und Anzeige der Tags zu Dokumenten eine Rolle. Je nach eingestellter Sprache auf der Webseite erstellt und sieht der Benutzer nur Tags, die mit dieser Sprache markiert sind.

Zum Zeitpunkt der Bearbeitunge dieser Arbeit befanden sich in der Datenbank der europäischen Spreadshirt-Plattform:

\begin{itemize}
    \item \num{2072079} Tags
    \item \num{6433410} Benutzer
    \item \num{26147860} Dokumente (\num{16494430} Artikel und \num{9653430} Designs)
    \item \num{76978414} Taggings
\end{itemize}

In der Menge der Tags befinden sich Tags in \num{15} verschiedenen Sprachen. Es wurden insgesamt \num{71936424} Dokumente mit Tags versehen.

\subsection{Clicktracking}

Spreadshirt betreibt ein Clicktracking-System, welches die Klicks der Benutzer auf Suchergebnisseiten aufzeichnet. Dabei ist unerheblich, ob der Benutzer bei Spreadshirt registriert und angemeldet ist. Dieses System sammelt Daten von beiden Spreadshirt-Plattformen (siehe \ref{platforms}) und erzeugt bei jedem Klick eines Besuchers auf ein Suchergebnis einen Datensatz mit folgenden Attributen:

\begin{itemize}
    \item Suchbegriff
    \item Plattform, \emph{EU} oder \emph{NA}
    \item Zeitstempel des Klicks
    \item ID des geklickten Dokumentes
    \item Position des geklickten Dokumentes auf der Ergebnisseite
    \item Sprache
\end{itemize}

Die Nutzung der Clicktracking-Daten liefert eine andere Sicht auf die Metadaten der Produkte als die Tags. Die Klicks liefern eine Einschätzung des suchenden Benutzers, ob die Metadaten, die für den Suchindex verwendet werden, zum Artikel selbst passen. Die Grundannahme ist hierbei, dass Benutzer nur auf Suchergebnisse klicken, die ihren Erwartungen bezüglich des Suchbegriffes gerecht werden.

Aufgrung der kürzlichen Einführung des Clicktracking-Systems wurde im Kontext dieser Arbeit mit den ersten \num{611836} aufgezeichneten Klicks gearbeitet.
\section{Lösungsansatz}